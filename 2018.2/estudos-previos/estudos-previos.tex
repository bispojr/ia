\documentclass[12pt,a4paper,oneside]{article}

\usepackage[utf8]{inputenc}
\usepackage[portuguese]{babel}
\usepackage[T1]{fontenc}
\usepackage{amsmath}
\usepackage{amsfonts}
\usepackage{amssymb}

\usepackage{multirow}
\usepackage{array,graphicx}

\usepackage{xcolor}
% Definindo novas cores
\definecolor{verde}{rgb}{0.25,0.5,0.35}
\definecolor{jpurple}{rgb}{0.5,0,0.35}

\author{\\Universidade Federal de Jataí (UFJ)\\Bacharelado em Ciência da Computação \\Inteligência Artificial - 2018.2 \\Prof. Esdras Lins Bispo Jr.}
\date{}

\title{
	\sc \huge Leituras para \\os Estudos Prévios
	\\{\tt Versão 1.0}
}

\begin{document}

\maketitle

\section{Livros de Referência}
	\begin{itemize}
		\item RUSSEL, S.; NORVIG, P. {\bf Inteligência Artificial}, 2ª Edição, Elsevier, 2004. \\ {\color{blue}{\bf Código Bib.: [004.89 RUS /int]}}.
		\item ARTERO, A. O. {\bf Inteligência Artificial}: Teoria e Prática, Editora Livraria da Física, 2009. \\ {\color{blue}{\bf Código Bib.: [004.89 ART /int]}}.
	\end{itemize}
	
\section{Trechos do Livro}

\begin{itemize}
	
	\item[] {\bf Aula 03 (17/08)}: Agentes Inteligentes
	\\Capítulo 2 (RUSSELL e NORVIG, 2004).
	
	\item[] {\bf Aula 05 (31/08)}: Resolução de Problemas por meio de Buscas
	\\ Capítulo 3 (RUSSELL e NORVIG, 2004).
	
	\item[] {\bf Aula 09 (14/09)}: Redes Neurais
	\\Seção 20.5 (RUSSELL e NORVIG, 2004).
	
	\item[] {\bf Aula 11 (15/09)}: Busca Local e Problemas de Otimização
	\\Seção 4.3 (RUSSELL e NORVIG, 2004).
	
	\item[] {\bf Aula 13 (21/09)}: Aprendizagem a partir de Exemplos
	\\Seções 18,1, 18.2 e 18.3 (RUSSELL e NORVIG, 2004).
	
	\item[] {\bf Aula 15 (28/09)}: Mineração de Dados
	\\ Seções 18.4 (RUSSELL e NORVIG, 2004) e 11.3 (ARTERO, 2009).
	
	\item[] {\bf Aula Extra (06/10)}: Agentes Lógicos
	\\ Seções 7.7 (RUSSELL e NORVIG, 2004) e 11.3 (ARTERO, 2009).
	
	\item[] {\bf Aula 25 (23/11)}: Outros tópicos
	\\Seções 22.1, 22.2 e 22.3 (RUSSELL e NORVIG, 2004).
	
\end{itemize}

\end{document}