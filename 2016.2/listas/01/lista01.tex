\documentclass[12pt,a4paper,oneside]{article}

\usepackage[utf8]{inputenc}
\usepackage[portuguese]{babel}
\usepackage[T1]{fontenc}
\usepackage{amsmath}
\usepackage{amsfonts}
\usepackage{amssymb}
\usepackage{graphicx}

\author{\\Universidade Federal de Goiás - UFG (Regional Jataí) \\Bacharelado em Ciência da Computação \\Inteligência Artificial \\Prof. Esdras Lins Bispo Jr.}

\title{
	{\sc \huge Lista de Exercícios 1} 
	\\{\tt Versão 2.0}
}

\begin{document}

\maketitle

\begin{enumerate}

	\item Leitura dos capítulos 1 e 2 (Russel e Norvig, 2013).
	
	\item {\bf [Russel 1.4]} Suponha que estendamos o programa ANALOGY de Evans para que possa alcançar 200 em um
teste de QI. Dessa forma teríamos um programa mais inteligente que um ser humano? Explique.	

	\item {\bf [Russel 1.7]} Até que ponto os sistemas seguintes são instâncias de inteligência artificial?
 		\begin{itemize}
 			\item Leitores de código de barra de supermercados.
			\item Menus de voz de telefones.
			\item Mecanismos de busca na Web.
			\item Algoritmos de roteamento da Internet que respondem dinamicamente ao estado da rede.
		\end{itemize}
	
	\item {\bf [Russel 1.11]} ``Sem dúvida, os computadores não podem ser inteligentes — eles só podem fazer o que seus programadores determinam.'' Esta última afirmação é verdadeira e implica a primeira?
	
	\item {\bf [Russel 1.12]} ``Sem dúvida, os animais não podem ser inteligentes — eles só podem fazer o que seus genes determinam.'' Esta última afirmação é verdadeira e implica a primeira?
	
	\item {\bf [Russel 1.13]} ``Sem dúvida, animais, seres humanos e computadores não podem ser inteligentes — eles só podem fazer o que seus átomos constituintes determinam, de acordo com as leis da física.'' Esta
última afirmação é verdadeira e implica a primeira?	
		
\end{enumerate}

\section{Referências}

\begin{itemize}
	\item RUSSELL, S.; NORVIG, P. Inteligência Artificial. Rio de Janeiro: Editora Campus, 2013.
\end{itemize}

\end{document}